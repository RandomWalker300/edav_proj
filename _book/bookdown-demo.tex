\documentclass[]{book}
\usepackage{lmodern}
\usepackage{amssymb,amsmath}
\usepackage{ifxetex,ifluatex}
\usepackage{fixltx2e} % provides \textsubscript
\ifnum 0\ifxetex 1\fi\ifluatex 1\fi=0 % if pdftex
  \usepackage[T1]{fontenc}
  \usepackage[utf8]{inputenc}
\else % if luatex or xelatex
  \ifxetex
    \usepackage{mathspec}
  \else
    \usepackage{fontspec}
  \fi
  \defaultfontfeatures{Ligatures=TeX,Scale=MatchLowercase}
\fi
% use upquote if available, for straight quotes in verbatim environments
\IfFileExists{upquote.sty}{\usepackage{upquote}}{}
% use microtype if available
\IfFileExists{microtype.sty}{%
\usepackage{microtype}
\UseMicrotypeSet[protrusion]{basicmath} % disable protrusion for tt fonts
}{}
\usepackage{hyperref}
\hypersetup{unicode=true,
            pdftitle={An edav project: hurricane analysis},
            pdfauthor={xx},
            pdfborder={0 0 0},
            breaklinks=true}
\urlstyle{same}  % don't use monospace font for urls
\usepackage{natbib}
\bibliographystyle{apalike}
\usepackage{longtable,booktabs}
\usepackage{graphicx,grffile}
\makeatletter
\def\maxwidth{\ifdim\Gin@nat@width>\linewidth\linewidth\else\Gin@nat@width\fi}
\def\maxheight{\ifdim\Gin@nat@height>\textheight\textheight\else\Gin@nat@height\fi}
\makeatother
% Scale images if necessary, so that they will not overflow the page
% margins by default, and it is still possible to overwrite the defaults
% using explicit options in \includegraphics[width, height, ...]{}
\setkeys{Gin}{width=\maxwidth,height=\maxheight,keepaspectratio}
\IfFileExists{parskip.sty}{%
\usepackage{parskip}
}{% else
\setlength{\parindent}{0pt}
\setlength{\parskip}{6pt plus 2pt minus 1pt}
}
\setlength{\emergencystretch}{3em}  % prevent overfull lines
\providecommand{\tightlist}{%
  \setlength{\itemsep}{0pt}\setlength{\parskip}{0pt}}
\setcounter{secnumdepth}{5}
% Redefines (sub)paragraphs to behave more like sections
\ifx\paragraph\undefined\else
\let\oldparagraph\paragraph
\renewcommand{\paragraph}[1]{\oldparagraph{#1}\mbox{}}
\fi
\ifx\subparagraph\undefined\else
\let\oldsubparagraph\subparagraph
\renewcommand{\subparagraph}[1]{\oldsubparagraph{#1}\mbox{}}
\fi

%%% Use protect on footnotes to avoid problems with footnotes in titles
\let\rmarkdownfootnote\footnote%
\def\footnote{\protect\rmarkdownfootnote}

%%% Change title format to be more compact
\usepackage{titling}

% Create subtitle command for use in maketitle
\providecommand{\subtitle}[1]{
  \posttitle{
    \begin{center}\large#1\end{center}
    }
}

\setlength{\droptitle}{-2em}

  \title{An edav project: hurricane analysis}
    \pretitle{\vspace{\droptitle}\centering\huge}
  \posttitle{\par}
    \author{xx}
    \preauthor{\centering\large\emph}
  \postauthor{\par}
      \predate{\centering\large\emph}
  \postdate{\par}
    \date{2019-10-30}

\usepackage{booktabs}
\usepackage{amsthm}
\makeatletter
\def\thm@space@setup{%
  \thm@preskip=8pt plus 2pt minus 4pt
  \thm@postskip=\thm@preskip
}
\makeatother

\begin{document}
\maketitle

{
\setcounter{tocdepth}{1}
\tableofcontents
}
\hypertarget{first}{%
\chapter{First}\label{first}}

This is a \emph{edav class} final project written in \textbf{Markdown}. we are working on it.

\hypertarget{intro}{%
\chapter{Introduction}\label{intro}}

explain why we chose this topic, and the questoins we are interested in studying.

we can write citation, for example, we are using the \textbf{bookdown} package \citep{R-bookdown} in this sample book, which was built on top of R Markdown and \textbf{knitr} \citep{xie2015}.

\hypertarget{methods}{%
\chapter{Methods}\label{methods}}

\hypertarget{data-sources}{%
\section{Data sources}\label{data-sources}}

We describe our data sources, our methods in this chapter.ion

\hypertarget{data-transformat}{%
\section{Data transformat}\label{data-transformat}}

Describe the process of getting the data into a form in which you could work with it in R.

\hypertarget{missing-values}{%
\section{Missing values}\label{missing-values}}

Describe any patterns you discover in missing values.

\hypertarget{results}{%
\chapter{Results}\label{results}}

Provide a short nontechnical but \emph{significant} summary of the most revealing findings of your analysis written for a nontechnical audience. Take extra care to clean up your graphs, ensuring that best practices for presentation are followed, as described in the audience ready style section below.

\hypertarget{discussion}{%
\chapter{Discussion}\label{discussion}}

Interactive component

Select one (or more) of your key findings to present in an interactive format. Be selective in the choices that you present to the user; the idea is that in 5-10 minutes, users should have a good sense of the question(s) that you are interested in and the trends you've identified in the data. In other words, they should understand the value of the analysis, be it business value, scientific value, general knowledge, etc.

\hypertarget{summary-and-conclusion}{%
\chapter{Summary and Conclusion}\label{summary-and-conclusion}}

Discuss limitations and future directions, lessons learned.

\bibliography{book.bib,packages.bib}


\end{document}
